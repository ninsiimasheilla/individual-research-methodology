

\documentclass[12pt,]{article}
\usepackage{zed-csp,graphicx,color}%from
\pagenumbering{roman}
\begin{document}

\begin{titlepage}
\centerline{effects of identity theft and deployment of biometrics to curb down identity theft in banks of uganda.\\}
\paragraph*{•}
\centerline{  Prepared by:  Ninsiima sheilla 16/U/964 216001139.\\}
\paragraph*{•}
\paragraph*{•}
  \begin{flushright}
  The Report,\\
  DATE: $February,6^{th},2018$.
 \tableofcontents

  \end{flushright}
\date{\today}
\end{titlepage}

\newpage





\pagenumbering{arabic}
\section{Introduction}
Identity theft is the use of another person’s identification such as credit card number, social security number or any other personal information to commit a crime or theft without the owner’s permission.
It causes a lot of damage to the victims by ruining their reputation and also consuming their money and time while resolving the issue.

\section{Statement of the problem}
The purpose of this study is to identify the effects of identity theft  to people and the use of biometrics systems to curb down the crime.

\section{Main objectives of the study}
To examine the effects and characteristics of identity theft and the common method used in stealing someone’s identity.
\section{specific objective of the study}

To know the different ways of how identity theft is done and alert fellow citizens to be careful and safeguard their identity.
To emphasize the installation of biometric system that will curb down the bad act of identity theft.
To determine if government officials claims and the medias portrayal of the substantial rise in identity theft incidents are supported empirically.


\section{Significance of the study}
To determine useful information to help understand and counter identity theft and identify how the thieves obtain false identities.
\section{Scope of the study}
This study was limited to only banks in Uganda where the effects and the ways on how to reduce identity theft were discussed.

\section{Methods of study used }
The data was obtained using analytical research where from the internet I managed to obtain some facts about identity theft and I was able to analyze its effects onto the victims and the different ways used to curb down the crime.
Quantitative methodology was used where I sampled a group of people in the bank and I found out that more than one hundred victims have been affected by identity theft.

\section{Recommendations }
Based on the findings and conclusions in this study, the following recommendations are made:

\begin{enumerate}

\item People should put a fraud alert on their credit reports. A fraud alert puts a red flag on your credit report and notifies lenders and creditors that they should take extra steps to verify your identity before extending credit.

\item People should contact any institution directly affected incase one’s credit or debit card has been stolen, report the theft to the credit card issuer or your bank.

\item People should regularly check credit reports and banking accounts through logging into their accounts atleast monthly to monitor activity and review your credit report atleast annually.
\end{enumerate}
\section{References}
\begin{itemize}
\item Graham Birley and Neil Moreland, 1998 A Practical Guide to Academic Research 
\item www.Creditsesame.com/blog/credit
\end{itemize}


\end{document}






